\documentclass[aspectratio=169]{beamer}

% ===================================
% Packages
% ===================================
\usepackage[utf8]{inputenc}
\usepackage[spanish]{babel}
\usepackage{graphicx}
\usepackage{tikz}
\usepackage{amssymb}
\usetikzlibrary{shapes.geometric, arrows.meta, positioning, backgrounds}

% ===================================
% Theme Configuration
% ===================================
\usetheme{Madrid}
\usecolortheme{default}

% Custom colors - Technological palette
\definecolor{primaryBlue}{RGB}{30, 58, 138}      % #1e3a8a
\definecolor{accentBlue}{RGB}{37, 99, 235}       % #2563eb
\definecolor{darkGray}{RGB}{31, 41, 55}          % #1f2937
\definecolor{mediumGray}{RGB}{107, 114, 128}     % #6b7280
\definecolor{lightGray}{RGB}{243, 244, 246}      % #f3f4f6

% Apply color scheme
\setbeamercolor{structure}{fg=primaryBlue}
\setbeamercolor{palette primary}{bg=primaryBlue,fg=white}
\setbeamercolor{palette secondary}{bg=accentBlue,fg=white}
\setbeamercolor{palette tertiary}{bg=darkGray,fg=white}
\setbeamercolor{frametitle}{bg=primaryBlue,fg=white}
\setbeamercolor{title}{fg=primaryBlue}
\setbeamercolor{block title}{bg=accentBlue,fg=white}
\setbeamercolor{block body}{bg=lightGray,fg=darkGray}
\setbeamercolor{item}{fg=accentBlue}

% Font configuration
\setbeamerfont{title}{size=\LARGE,series=\bfseries}
\setbeamerfont{frametitle}{size=\Large,series=\bfseries}
\setbeamerfont{framesubtitle}{size=\small}

% Remove navigation symbols
\setbeamertemplate{navigation symbols}{}

% Customize itemize
\setbeamertemplate{itemize items}[circle]

% Footer
\setbeamertemplate{footline}{
  \leavevmode%
  \hbox{%
  \begin{beamercolorbox}[wd=.33\paperwidth,ht=2.5ex,dp=1ex,center]{palette tertiary}%
    \usebeamerfont{author in head/foot}\insertshortauthor
  \end{beamercolorbox}%
  \begin{beamercolorbox}[wd=.34\paperwidth,ht=2.5ex,dp=1ex,center]{palette secondary}%
    \usebeamerfont{title in head/foot}\insertshorttitle
  \end{beamercolorbox}%
  \begin{beamercolorbox}[wd=.33\paperwidth,ht=2.5ex,dp=1ex,right]{palette tertiary}%
    \usebeamerfont{date in head/foot}\insertshortdate{}\hspace*{2em}
    \insertframenumber{} / \inserttotalframenumber\hspace*{2ex}
  \end{beamercolorbox}}%
  \vskip0pt%
}

% ===================================
% Title Information
% ===================================
\title{Justificación para la adquisición de una tablet con lápiz digital}
\subtitle{Herramienta académica para estudios universitarios}
\author{Jefferson Franco}
\institute{Ingeniería de Sistemas • Ingeniería Informática}
\date{2026}

% ===================================
% Document
% ===================================
\begin{document}

% ===================================
% Title Slide
% ===================================
\begin{frame}
  \titlepage
\end{frame}

% ===================================
% Contexto Académico
% ===================================
\begin{frame}{Contexto Académico}
  \begin{block}{Carga académica elevada}
    \begin{itemize}
      \item Dos carreras universitarias simultáneas
      \item Ingeniería de Sistemas e Ingeniería Informática
      \item Alta demanda de organización y gestión del tiempo
    \end{itemize}
  \end{block}
  
  \vspace{0.5cm}
  
  \begin{center}
    \begin{tikzpicture}
      \node[rectangle, rounded corners, fill=accentBlue, text=white, minimum width=8cm, minimum height=1.2cm] 
        {\Large\textbf{10 cursos en el semestre actual}};
    \end{tikzpicture}
  \end{center}
  
  \vspace{0.3cm}
  
  \begin{itemize}
    \item Alta demanda técnica y matemática
    \item Requiere metodología de estudio eficiente
  \end{itemize}
\end{frame}

% ===================================
% Magnitud del Semestre
% ===================================
\begin{frame}{Magnitud del Semestre}
  \begin{columns}[T]
    \begin{column}{0.48\textwidth}
      \begin{block}{Cursos Técnicos (6)}
        \begin{itemize}
          \item Algoritmos
          \item Programación Competitiva
          \item Estructuras de Datos Avanzados
          \item Investigación de Operaciones
          \item Sistemas de Comunicación
          \item Sistemas de Información
        \end{itemize}
      \end{block}
    \end{column}
    
    \begin{column}{0.48\textwidth}
      \begin{block}{Cursos de Educación (4)}
        \begin{itemize}
          \item Estadística
          \item Sistemas Numéricos
          \item Cursos complementarios
          \item Seguimiento continuo
        \end{itemize}
      \end{block}
    \end{column}
  \end{columns}
  
  \vspace{0.5cm}
  \centering
  \textcolor{primaryBlue}{\textbf{Cada curso requiere apuntes, diagramas y resolución de problemas}}
\end{frame}

% ===================================
% Algoritmos
% ===================================
\begin{frame}{Algoritmos}
  \begin{itemize}
    \item Análisis y diseño de algoritmos complejos
    \item Matemáticas aplicadas (complejidad, optimización)
    \item Diagramas de flujo y pseudocódigo
    \item Corrección y optimización iterativa
  \end{itemize}
  
  \vspace{0.5cm}
  
  \begin{center}
    \begin{tikzpicture}[node distance=1.5cm]
      \node (start) [rectangle, rounded corners, fill=accentBlue!20, text width=2.5cm, text centered] {Problema};
      \node (analyze) [rectangle, rounded corners, fill=accentBlue!40, text width=2.5cm, text centered, right=of start] {Análisis};
      \node (design) [rectangle, rounded corners, fill=accentBlue!60, text width=2.5cm, text centered, right=of analyze] {Diseño};
      \node (optimize) [rectangle, rounded corners, fill=accentBlue!80, text width=2.5cm, text centered, right=of design] {Optimización};
      
      \draw[-{Stealth[length=3mm]}] (start) -- (analyze);
      \draw[-{Stealth[length=3mm]}] (analyze) -- (design);
      \draw[-{Stealth[length=3mm]}] (design) -- (optimize);
    \end{tikzpicture}
  \end{center}
  
  \vspace{0.3cm}
  \centering
  \textcolor{mediumGray}{\small Requiere corrección constante y esquemas visuales}
\end{frame}

% ===================================
% Programación Competitiva
% ===================================
\begin{frame}{Programación Competitiva}
  \begin{itemize}
    \item Resolución intensiva de problemas algorítmicos
    \item Práctica constante con retroalimentación inmediata
    \item Necesidad de corrección y refinamiento continuo
    \item Documentación de estrategias y soluciones
  \end{itemize}
  
  \vspace{0.5cm}
  
  \begin{center}
    \begin{tikzpicture}
      \foreach \x/\label in {1/Problema, 2/Solución, 3/Corrección, 4/Optimización} {
        \node[circle, fill=accentBlue, text=white, minimum size=1.8cm] at ({(\x-1)*3},0) {\textbf{\x}};
        \node[text=darkGray] at ({(\x-1)*3},-1) {\label};
      }
      \foreach \x in {1,2,3} {
        \draw[-{Stealth[length=3mm]}, very thick, accentBlue] ({(\x-1)*3+0.9},0) -- ({(\x)*3-0.9},0);
      }
    \end{tikzpicture}
  \end{center}
\end{frame}

% ===================================
% Estructuras de Datos Avanzados
% ===================================
\begin{frame}{Estructuras de Datos Avanzados}
  \begin{itemize}
    \item Árboles binarios, AVL, B-trees
    \item Grafos y algoritmos de recorrido
    \item Estructuras complejas y sus aplicaciones
    \item Importancia crítica de esquemas visuales
  \end{itemize}
  
  \vspace{0.5cm}
  
  \begin{center}
    \begin{tikzpicture}[
      level 1/.style={sibling distance=3cm},
      level 2/.style={sibling distance=1.5cm},
      every node/.style={circle, draw=accentBlue, fill=accentBlue!20, minimum size=0.8cm}
    ]
      \node {R}
        child {node {A}
          child {node {C}}
          child {node {D}}
        }
        child {node {B}
          child {node {E}}
          child {node {F}}
        };
    \end{tikzpicture}
  \end{center}
  
  \vspace{0.3cm}
  \centering
  \textcolor{mediumGray}{\small Los diagramas son esenciales para la comprensión}
\end{frame}

% ===================================
% Investigación de Operaciones
% ===================================
\begin{frame}{Investigación de Operaciones}
  \begin{itemize}
    \item Modelos matemáticos extensos y complejos
    \item Programación lineal y optimización
    \item Procedimientos paso a paso detallados
    \item Múltiples iteraciones y correcciones
  \end{itemize}
  
  \vspace{0.5cm}
  
  \begin{block}{Ejemplo: Método Simplex}
    \centering
    \begin{tikzpicture}
      \node[rectangle, rounded corners, fill=lightGray, text=darkGray, text width=10cm, align=center] {
        Requiere tablas extensas, cálculos iterativos\\
        y correcciones frecuentes durante el proceso
      };
    \end{tikzpicture}
  \end{block}
  
  \vspace{0.3cm}
  \centering
  \textcolor{primaryBlue}{\textbf{Imposible de gestionar eficientemente en papel}}
\end{frame}

% ===================================
% Sistemas de Comunicación de Datos
% ===================================
\begin{frame}{Sistemas de Comunicación de Datos}
  \begin{itemize}
    \item Arquitecturas de redes y protocolos
    \item Diagramas de topologías de red
    \item Modelos OSI y TCP/IP
    \item Esquemas de comunicación y flujo de datos
  \end{itemize}
  
  \vspace{0.5cm}
  
  \begin{center}
    \begin{tikzpicture}
      \foreach \y/\label in {0/Aplicación, 1/Transporte, 2/Red, 3/Enlace} {
        \node[rectangle, rounded corners, fill=accentBlue, text=white, minimum width=6cm, minimum height=0.8cm] 
          at (0,{-\y*1}) {\label};
      }
    \end{tikzpicture}
  \end{center}
  
  \vspace{0.3cm}
  \centering
  \textcolor{mediumGray}{\small Contenido altamente visual y esquemático}
\end{frame}

% ===================================
% Sistemas de Información e Ingeniería de Procesos
% ===================================
\begin{frame}{Sistemas de Información e Ingeniería de Procesos}
  \begin{itemize}
    \item Diagramas de procesos de negocio
    \item Modelado de sistemas empresariales
    \item Diagramas UML y de flujo
    \item Documentación técnica extensa
  \end{itemize}
  
  \vspace{0.5cm}
  
  \begin{center}
    \begin{tikzpicture}[node distance=2cm]
      \node (req) [rectangle, rounded corners, fill=accentBlue!30, minimum width=2cm, minimum height=1cm] {Requisitos};
      \node (design) [rectangle, rounded corners, fill=accentBlue!50, minimum width=2cm, minimum height=1cm, right=of req] {Diseño};
      \node (impl) [rectangle, rounded corners, fill=accentBlue!70, minimum width=2cm, minimum height=1cm, right=of design] {Implementación};
      
      \draw[-{Stealth[length=3mm]}, very thick] (req) -- (design);
      \draw[-{Stealth[length=3mm]}, very thick] (design) -- (impl);
    \end{tikzpicture}
  \end{center}
  
  \vspace{0.3cm}
  \centering
  \textcolor{primaryBlue}{\textbf{Requiere herramientas de diagramación eficientes}}
\end{frame}

% ===================================
% Cursos de Educación
% ===================================
\begin{frame}{Cursos de Educación}
  \begin{block}{Estadística}
    \begin{itemize}
      \item Análisis de datos y distribuciones
      \item Gráficos y representaciones visuales
      \item Cálculos extensos y tablas
    \end{itemize}
  \end{block}
  
  \vspace{0.3cm}
  
  \begin{block}{Sistemas Numéricos}
    \begin{itemize}
      \item Conversiones entre bases numéricas
      \item Operaciones y algoritmos
      \item Registro y seguimiento detallado
    \end{itemize}
  \end{block}
  
  \vspace{0.3cm}
  \centering
  \textcolor{mediumGray}{\small Requieren organización y claridad en los apuntes}
\end{frame}

% ===================================
% Problema del Método Tradicional
% ===================================
\begin{frame}{Problema del Método Tradicional}
  \begin{columns}[T]
    \begin{column}{0.48\textwidth}
      \begin{block}{Situación Actual}
        \begin{itemize}
          \item 10 cuadernos diferentes
          \item Mochila excesivamente pesada
          \item Desorganización frecuente
          \item Dificultad para encontrar información
          \item Imposibilidad de corrección eficiente
        \end{itemize}
      \end{block}
    \end{column}
    
    \begin{column}{0.48\textwidth}
      \begin{center}
        \vspace{2cm}
        {\Huge\textcolor{red}{\textbf{INEFICIENTE}}}
        \vspace{0.5cm}
        
        \textcolor{darkGray}{Método tradicional\\no sostenible}
      \end{center}
    \end{column}
  \end{columns}
\end{frame}

% ===================================
% La Tablet como Solución
% ===================================
\begin{frame}{La Tablet como Solución}
  \begin{center}
    \vspace{0.5cm}
    {\Huge\textcolor{green!60!black}{\textbf{SOLUCIÓN ÓPTIMA}}}
    \vspace{0.5cm}
  \end{center}
  
  \begin{itemize}
    \item \textbf{Un solo dispositivo} para todos los cursos
    \item \textbf{Apuntes digitales} con lápiz táctil de precisión
    \item \textbf{Organización centralizada} por carpetas y etiquetas
    \item \textbf{Búsqueda instantánea} de información
    \item \textbf{Corrección sin límites} (borrar y reescribir)
    \item \textbf{Sincronización en la nube} (respaldo automático)
  \end{itemize}
\end{frame}

% ===================================
% Beneficios Académicos
% ===================================
\begin{frame}{Beneficios Académicos}
  \begin{block}{Organización}
    \begin{itemize}
      \item Todos los apuntes en un solo lugar
      \item Clasificación por curso, tema y fecha
      \item Acceso inmediato a cualquier contenido
    \end{itemize}
  \end{block}
  
  \begin{block}{Eficiencia en el Estudio}
    \begin{itemize}
      \item Repasos más rápidos y efectivos
      \item Posibilidad de anotar sobre PDFs y diapositivas
      \item Integración con herramientas académicas
    \end{itemize}
  \end{block}
  
  \begin{block}{Calidad}
    \begin{itemize}
      \item Diagramas más claros y precisos
      \item Corrección ilimitada sin desperdiciar papel
      \item Mayor claridad visual en apuntes
    \end{itemize}
  \end{block}
\end{frame}

% ===================================
% Beneficios para la Salud
% ===================================
\begin{frame}{Beneficios para la Salud}
  \begin{columns}[T]
    \begin{column}{0.48\textwidth}
      \begin{block}{Reducción de Peso}
        \begin{itemize}
          \item Mochila significativamente más ligera
          \item Menor carga en espalda y hombros
          \item Prevención de problemas posturales
        \end{itemize}
      \end{block}
    \end{column}
    
    \begin{column}{0.48\textwidth}
      \begin{block}{Mejor Postura}
        \begin{itemize}
          \item Dispositivo ergonómico
          \item Ajuste de ángulo de visualización
          \item Reducción de fatiga física
        \end{itemize}
      \end{block}
    \end{column}
  \end{columns}
  
  \vspace{0.5cm}
  
  \begin{center}
    \begin{tikzpicture}
      \node[rectangle, rounded corners, fill=green!20, text=darkGray, minimum width=10cm, minimum height=1.2cm, align=center] 
        {\textbf{Impacto positivo en salud física a largo plazo}};
    \end{tikzpicture}
  \end{center}
\end{frame}

% ===================================
% Inversión Educativa
% ===================================
\begin{frame}{Inversión Educativa}
  \begin{itemize}
    \item Costo razonable en comparación con beneficios
    \item Uso a largo plazo (toda la carrera universitaria)
    \item Herramienta profesional para el futuro
    \item Retorno de inversión en productividad académica
  \end{itemize}
  
  \vspace{0.5cm}
  
  \begin{center}
    \begin{tikzpicture}
      \node[rectangle, rounded corners, fill=accentBlue, text=white, minimum width=8cm, minimum height=2.5cm, align=center] 
        {\Large\textbf{No es un gasto}\\[0.3cm]
         \large Es una inversión en educación\\
         y desarrollo profesional};
    \end{tikzpicture}
  \end{center}
\end{frame}

% ===================================
% Compromiso Personal
% ===================================
\begin{frame}{Compromiso Personal}
  \begin{block}{Uso Académico}
    \begin{itemize}
      \item Dedicación exclusiva a actividades universitarias
      \item Herramienta de trabajo, no entretenimiento
      \item Enfoque en productividad y organización
    \end{itemize}
  \end{block}
  
  \vspace{0.3cm}
  
  \begin{block}{Responsabilidad}
    \begin{itemize}
      \item Cuidado y mantenimiento del dispositivo
      \item Uso eficiente y planificado
      \item Aprovechamiento máximo de sus capacidades
    \end{itemize}
  \end{block}
  
  \vspace{0.3cm}
  
  \begin{center}
    \textcolor{primaryBlue}{\large\textbf{Compromiso con la excelencia académica}}
  \end{center}
\end{frame}

% ===================================
% Conclusión
% ===================================
\begin{frame}{Conclusión}
  \begin{center}
    \Large\textcolor{primaryBlue}{\textbf{No es un lujo, es una herramienta necesaria}}
  \end{center}
  
  \vspace{0.8cm}
  
  \begin{itemize}
    \item La carga académica actual requiere herramientas eficientes
    \item El método tradicional es insostenible para 10 cursos técnicos
    \item La tablet optimiza organización, salud y rendimiento
    \item Inversión educativa con beneficios a largo plazo
    \item Compromiso personal de uso responsable y académico
  \end{itemize}
  
  \vspace{0.8cm}
  
  \begin{center}
    \begin{tikzpicture}
      \node[rectangle, rounded corners, fill=primaryBlue, text=white, minimum width=10cm, minimum height=1.5cm, align=center] 
        {\Large\textbf{Decisión planificada y racional}};
    \end{tikzpicture}
  \end{center}
\end{frame}

% ===================================
% Final Slide
% ===================================
\begin{frame}
  \begin{center}
    \Huge\textcolor{primaryBlue}{\textbf{Gracias}}
    
    \vspace{1cm}
    
    \Large Jefferson Franco
    
    \vspace{0.3cm}
    
    \large Ingeniería de Sistemas • Ingeniería Informática
    
    \vspace{0.3cm}
    
    \normalsize 2026
  \end{center}
\end{frame}

\end{document}
